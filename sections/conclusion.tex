\chapter{Conclusion}\label{ch:conclusion}

%The communication between the two nodes is established by "UI\_turtlesim\_mover" and "turtlesim\_mover". These nodes talk to eachother by sending strings of information between them by topics.\\
%To conclude, the code has met the standards which was set from the start. \\
%The code will execute the commands send in by the user and show exactly how it works by demonstrating the message in turtlesim.

The aim for this mini-project has been to focus on communication of the nodes and what they publish.\\ In the process of building the nodes, for manipulating the turtlesim, it gave a clear understanding of the interactions between the nodes in the ROS system.\\
The nodes was built in a way so the user-interface, would have control over what had to be published. By using the \texttt{topic list} we could clearly see which topics were being used in the turtlesim, when we listened to these topics while moving the turtle, it was possible to determine which data were published. We could determine that the turtlesim\_mover needed to send a set of floats in a message to the turtlesim to make it move. While the turtlesim\_mover was connected to the turtlesim, it would process the messages and visualization, and what has been written in the codes will affect the turtlesim program.\\
