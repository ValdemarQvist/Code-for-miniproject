\chapter{Conclusion}\label{ch:conclusion}

%The communication between the two nodes is established by "UI\_turtlesim\_mover" and "turtlesim\_mover". These nodes talk to eachother by sending strings of information between them by topics.\\
%To conclude, the code has met the standards which was set from the start. \\
%The code will execute the commands send in by the user and show exactly how it works by demonstrating the message in turtlesim.

The aim for this mini-project has been to focus on communication of the nodes and what they publish.\\ In the process of building the nodes, for manipulating the turtlesim, gave a clear understanding of the interactions between nodes in the ROS system.\\
The nodes was build in a way that the user-interface, would have control over what needed to be published. By using the \texttt{topic list} we could clearly see what topics was used in the turtlesim, when we listened to these topics while moving the turtle it was possible to determine what data was published. We could determine that the turtlesim\_mover needed to send a string of commands to the turtlesim. While the listener was connected to turtlesim, it would then process the message and visualize what has been written in the terminal will affect a third-party program.\\
