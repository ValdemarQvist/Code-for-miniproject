\chapter{Introduction}

\section{Introduction to ROS}
ROS or Robot Operation System was developed by Willows garage in 2007, but now it is open source. ROS is a framework to interact with robots, it is a compilation of tools, libraries, and different environments\cite{ROSINTRO}. ROS is using nodes which communicate with each other either back and forward or one-way, the way they do this is by publishing or subscribing to topics. One down side to publishing is that there are no feed back, so the publisher does not know whether the subscriber is receiving the messages which was sent.\\
In this mini-project there will be a demonstration of how this can be setup.

\section{Turtlesim}

The turtlesim is a simulator that was developed by ROS to get a better understanding of nodes, topics and visualizing the communication between the nodes.\\
In the simulator a turtle can be manipulated in a closed environment. One thing to know about the turtlesim space is that it works on a bi-dimensional plane, where the position of the turtle is given by a x and an y coordinate and the heading is given by the angle theta.

\section{Introduction of nodes}\label{ch:introduction}


In this section there will be an introduction to how the 2 nodes are functioning. 

\subsection{UI\_Turtlesim\_mover node}

The UI\_Turtlesim\_mover node, is the publisher to the Turtlesim\_mover node.\\
This program will ask whether you want the Turtlesim to draw a circle, square or a star.\\
This is where the user can press 1,2,3 accordingly to what type of object the user would like to be drawn. This will then send a topic to the subscriber, with the information about the choice which the user made.\\


\subsection{Turtlesim\_mover node}

The Turtlesim\_mover node is the node which make the turtlesim move in a square, circle or random movements.\\
It is listening to the topic, which the UI is publishing on when the user picks a choice for the turtlesim.
It will send a message back to the UI node and confirm that it needs something to do or that it is doing the command of the user.\\